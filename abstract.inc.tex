    This paper/poster describes experiences tagging the Hansard corpus using the CLAWS and USAS semantic tagger on the Lancaster University high performance cluster.  Herein we describe how we were able to parallelise and apply a ``traditional'' single-threaded batch-oriented application to a platform that differs greatly from that for which it was originally designed.\\
    We start by discussing the tagging toolchain, its specific requirements and properties, and its performance characteristics.  This is contrasted with a description of the cluster on which it was to run, and specific limitations are discussed such as the overhead of using SAN-based storage.\\
    We then go on to discuss the nature of the Hansard corpus, and describe which properties of this corpus in particular prove challenging for use on the system architecture we used.\\
    The solution for tagging the corpus is then described, along with performance comparisons against a naive run on commodity hardware.  We discuss the gains and benefits of using high-performance machinery rather than relatively cheap commodity hardware.
