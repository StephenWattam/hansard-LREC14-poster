\subsection{Limitations \& Solutions}
\begin{itemize}
    \item Small files (filesystem overhead)
        \begin{itemize}
            \item Input files [tarred, stored in RO storage]
            \item Temp files and toolchain comms [on nodes]
            \item Output file storage [tarred, stored on shared disk after tarring]
        \end{itemize}
    \item Small jobs (job overhead)
        \begin{itemize}
            \item Batching using ID numbers
            \item Scheduling to distribute load (many jobs, composite job thing)
        \end{itemize}
    \item Co-ordination of parallel tasks
        \begin{itemize}
            \item Use of a pre-built index
            \item batches of fixed size, job ID computes offset
        \end{itemize}
\end{itemize}



\subsection{Final Method}
\begin{itemize}
    \item Corpus pre-tarred into daily runs so as to avoid filesize issues and problems with multiple-directory structure
    \item Corpus stored in read-only space due to size
    \item Everything compiled in ~/.local directory on server, batch scripts set up for toolchain in new shell
    \item Produce directory listing of input file groups
    \item Batch file
        \begin{itemize}
            \item Computes batch size from batch ID
            \item Duplicates directory tree in the working compute node
            \item extracts everything
            \item Runs pipeline over each file
            \item Archives results 
            \item Archives all of those results to produce a large file for storage on panasas shelf
            \item Copies back
            \item Deletes compute node
            \item Stores stdout/stderr in common location with batch ID
        \end{itemize}
    \item Output structure taken off HEC
    \item Validation script tests:
        \begin{itemize}
            \item output files exist
            \item The right number of files is in the output/input tars
            \item File sizes are nonzero
        \end{itemize}
\end{itemize}



