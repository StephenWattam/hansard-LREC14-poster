Requirements for NLP systems now typically include the need for extremely large scale processing of data derived either from vast online sources or increasing quantities of digitised archive material. 
Whether the scenario is to answer a particular set of research questions or for commercial text analytics, sources such as Twitter provide significantly more data than can be processed and ingested in anything like reasonable time. Parallel computation is typically used to address this bottleneck.

This poster presents the lessons learnt from work which emerged from two serendipitous events: the need to support digital humanists in their analysis of the two-billion-word Hansard data set, and a need for a real case study for Lancaster University's High Performance Cluster using textual rather than numerical data. 
Large infrastructure activities such as CLARIN~\cite{varadi2008clarin} and DARIAH~\cite{constantopoulos2008preparing} are providing distributed archives for language resources but NLP researchers still face local requirements during experimentation to process
% and reprocess 
very large resources through complex pipelines. 
Some toolkits, e.g. GATE, can now run in the cloud~\cite{tablan2013gatecloud} to support such activities. However, many universities have existing high performance clusters that may be under--exploited by language researchers.  

The work described here is part of a larger project, Parliamentary Discourse%
\footnote{\url{http://www.glasgow.ac.uk/hansard}}%
, to include the 200-year corpus of Hansard texts in Enroller%
\footnote{\url{http://www.glasgow.ac.uk/enroller} Enroller contains five corpora, one dictionary and one thesaurus. The data sets can be simultaneously searched and the 780,000-entry Historical Thesaurus of English enables searching of textual resources by concepts and across time.}%
, an infrastructure to hold and cross-search large data sets. The wider aim was to enrich the Hansard corpus with linguistic annotations and named-entity markup, cross--link it to the resources within the Enroller portal and make the resource available for linguistic and historical research. 
Our poster provides a valuable scenario for large scale NLP pipelines and lessons learnt from the experience.
% \dr{Todo: signposting}
